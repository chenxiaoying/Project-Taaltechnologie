\documentclass[a4paper,11pt]{article}
\usepackage[english]{babel}
\usepackage[round,longnamesfirst]{natbib}

\title{Report Language Technology}
\author{Roald Baas \& Xiaoying Chen \& Anne Wanningen\\(group 13)}
%\date{ }

\begin{document}

%The report should be approx. 5 pages and should ideally address the following topics:

%    A description of dbpedia
%    A description of the task (question answering, what kind of questions)
%    A description of the architecture and major componets of your QA-system
%    Additional knowledge resources used (if any)
%    Results on the test questions
%    Error analysis
%    Division of work in your team (who contributed to what)
%    Each group only submits a single report

\maketitle

\section{Introduction}
% Description of the task
% Description of dbpedia (use of dbpedia to do task), dbpedia citation?
% Next, we describe the architecture of the system



\section{Architecture}
% Use of pairCounts + similarwords
%TODO: alpino citation? xpath citation?
The system was built in Python3 (http://www.python.org) using the libraries lxml and SPARQLWrapper. We make use of the files \emph{pairCounts} and \emph{similarwords} provided by Spotlight \citep{isem2013daiber} to determine dbpedia URIs and properties of questions. Alpino is used to parse questions, on which in turn xpath is used to extract keywords.

%TODO: SPARQL gedeelte

% Major components, how everything works, xpath SPARQL etc.

\subsection{Method 1}
% First faster method


\subsection{Method 2}
% Second slower method



\section{Results}
% Results
% 41 questions answered, 15 correct. Precision, recall, etc.

\subsection{Error analysis}
% What went wrong




\bibliographystyle{plainnat}
\bibliography{literature}


\end{document}
